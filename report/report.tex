\documentclass[12pt]{article}

\usepackage[french]{babel}
\usepackage[utf8]{inputenc}  
\usepackage[T1]{fontenc}
\usepackage[left=2.5cm,right=2.5cm,top=3cm,bottom=2.65cm]{geometry}
\usepackage{amsmath,amssymb,amsfonts}
\usepackage{algorithm}
\usepackage{algpseudocode}
\usepackage{graphicx}
\usepackage{caption}
\usepackage{tikz}

\graphicspath{{ims/}}

\title{
	Distributed Systems\\
	\textbf{Building a distributed neural network}\\
	Report
}
\author{
	Yoann Coudert-\,-Osmont
}

\begin{document}
	
\maketitle

\section{Introduction}

In this project a neural network for distributed system has been implemented. We use an architecture with fully connected layers. Each neuron of the network is a single process which communicate with its in-neurons and out-neurons. The user can spawn a monitor which will create the network an manage it.

\paragraph{How to use the code}
The source code is in the folder \verb|src|. There are two files which were given to us at the beginning of the project (\verb|dataframe.erl| and \verb|utils.erl|) to make easier the use of 2d arrays and to read CSV files.
\begin{itemize}
	\item The file \verb|network.erl| contains functions executed by neurons and a function to spawn a network.
	\item The file \verb|monitor.erl| contains the code of the monitor.
	\item The file \verb|testing.erl| allows to test the efficiency of different architectures of neural networks.
\end{itemize}
The folder \verb|testing| contains some Python script to plot some graphics which are in this report. The folder \verb|data| contains two data-sets in CSV format. \\
To execute the code you can use:
\begin{itemize}
	\item \verb|make network| to test feed-forward, feed-backward and interruption of these scenarios. More precisely it tests the interruption of a feed-forward scenario at the begging. Next it try to make the network output 1 on the input (0.5, 0.4). It print at each step two lines with \verb|res = done| for the feed-backward scenario and \verb|res = <a float>| for the result of the feed-forward scenario. At the end it tries to interrupt a feed-backward scenario.
	\item \verb|make monitor| to test the monitor which tests backpropagation algorithm on the data-set in the file \verb|training_set.csv|.
	\item \verb=make testing | python3 testing/stats.py= to plot some graphs about the efficiency of different neural network architectures.
\end{itemize}

\paragraph{Parameters}
At the beginning of the file \verb|src/network.erl| there is a section parameters in which you can:
\begin{itemize}
	\item Choose the time before re-sending a message when no ack has been received.
	\item Choose the learning rate.
	\item Activate the verbose mode.
	\item Choose the probability of failure when sending a message between two neurons.
\end{itemize}

\section{Network deployment}

I spawn all neurons on the same node. I didn't implemented a way to spawn layers on different nodes. The function that spawn a network is \verb|network:neural_network/2|. the first parameter is a list of the sizes of each layer. The second is the PID of the monitor with which output and input neurons will communicate. I use a trick which consist in spawning a new input neurons before the true layer of inputs neurons. This new neurons will have a list of values to feed to the true input layer in feed-forward scenario.

\begin{figure}[h]
	\centering
	\begin{tikzpicture}[every node/.style={minimum size=0.75cm}]
		\node[fill=blue, draw, circle, opacity=0.5] (a) at (0, 0) {};
		
		\node[fill=green, draw, circle, opacity=0.5] (b) at (2.5, -1.4) {};
		\node[fill=green, draw, circle, opacity=0.5] (c) at (2.5, 0) {};
		\node[fill=green, draw, circle, opacity=0.5] (d) at (2.5, 1.4) {};
		
		\node[fill=gray, draw, circle, opacity=0.5] (e) at (5, -0.7) {};
		\node[fill=gray, draw, circle, opacity=0.5] (f) at (5, 0.7) {};
		
		\node[fill=gray, draw, circle, opacity=0.5] (g) at (7.5, -0.7) {};
		\node[fill=gray, draw, circle, opacity=0.5] (h) at (7.5, 0.7) {};
		
		\node[fill=red, draw, circle, opacity=0.5] (i) at (10, 0) {};
		
		\node[fill=black, draw, circle, opacity=0.5] (j) at (5, 3) {};
		\draw[<->] (a) edge[bend left] (j);
		\draw[<->] (i) edge[bend right] (j);
		
		\node at (6.25, -1.8) {\small Hidden layers};
		\node at (2.5, -2.4) {\small Input layer};
		\node at (-0.3, -1) {\small Input neuron};
		\node at (10.3, -1) {\small Output neuron};
		\node at (5, 3.75) {\small Monitor};
		
		\draw[<->] (a) -- (b);
		\draw[<->] (a) -- (c);
		\draw[<->] (a) -- (d);
		
		\draw[<->] (b) -- (e);
		\draw[<->] (b) -- (f);
		\draw[<->] (c) -- (e);
		\draw[<->] (c) -- (f);
		\draw[<->] (d) -- (e);
		\draw[<->] (d) -- (f);
		
		\draw[<->] (e) -- (g);
		\draw[<->] (e) -- (h);
		\draw[<->] (f) -- (g);
		\draw[<->] (f) -- (h);
		
		\draw[<->] (g) -- (i);
		\draw[<->] (h) -- (i);
	\end{tikzpicture}
	\captionsetup{justification=centering}
	\caption{All processes spawned and links between them. The first argument of
		\texttt{network:neural\_network/2} was \texttt{[3, 2, 2, 1]}. }
\end{figure}

Layers are spawn one after the other starting from the layer of the output neuron. Neurons are spawn with their out-layer in parameter then they are waiting for a message containing their in-layer. When the in-layer is spawned we send it.

I have three different functions for the neuron. A function for hidden neurons and for the neurons of the true input layer, a function for the input neuron and a function for the output neuron.

\section{Feed-forward and feed-backward scenarios}

\paragraph{Weights and Bias}
As proposed weights and bias are initialized with random values $\sim \mathcal{N}(0, 1)$. I created a record \verb|hd| that contains data of hidden neurons (When I say hidden neuron, I include neurons of the input layer. I only distinguish the input neuron and the output neuron from other neurons). In this record I store the weights of connections toward the out-layer (i.e. the values $w_{ij}^{l+1}$ for the neuron $j$ of the layer $l$ if use the notations of the subject). I preferred store these weights instead of weights of the connections with in-layer $w_{jk}^l$ as proposed in the subject because it allow us to avoid the need to store the values of in-neuron to update weights during the feed-backward scenario. In fact we have:
$$ w_{ij}^{l+1} \gets w_{ij}^{l+1} - \eta a_j^l \delta_i^{l+1} $$
Then the $j$-th neuron of the $l$-th layer has to send weighted values during feed-forward scenario (i.e. it sends $w_{ij}^{l+1} a_j^l$ to the $i$-th neuron of the $(l+1)$-th layer).

I also store the bias $b_j^l$ in \verb|hd|. As neurons of the input layer should not have a bias, the record also have a value \verb|input_layer| which is \verb|true| if and only if the neuron is a neuron of the input layer. When this value is \verb|true|, the bias $b$ is forced to be 0.

Furthermore the output neuron stores its own bias.

\paragraph{Communications}
I created another record \verb|com| for communications. The three parameters of the hidden neuron function are a record \verb|hd| containing the data of the neuron, and two record \verb|com|, one for the communication of the feed-forward scenario and another for the communication of the feed-backward scenario.

Now I will explain how communications works in the case of feed-forward scenario. I use the idea of the alternative-bit algorithm. In the \verb|com| record I have the following values:
\begin{itemize}
	\item \verb|rcv|, an array which store a bit for all neuron from which we will receive messages (i.e. neurons of the in-layer in the case of feed-forward scenario).
	\item \verb|send|, an array which store a bit for all neuron to which we will send messages (i.e. neurons of the out-layer in the case of feed-forward scenario).
	\item \verb|miss_rcv|, the number of messages that we need to receive.
	\item \verb|miss_send|, the number of ack that we need to receive.
	\item \verb|timeout|, the time in nano-seconds of the last message sent.
	\item \verb|bit|, the alternative-bit that represent the state of the communication.
\end{itemize}
The two arrays \verb|rcv| and \verb|send| are initialized with zeros. \verb|bit| is also initialized to zero. Finally the first state of a neuron is waiting for data so \verb|miss_rcv| is initialized to the length of \verb|rcv| (\verb|size(rcv)|) and \verb|miss_send| is initialized to zero.

We can describe the protocol with three labels. Two labels for receiving messages and one label for re-sending messages for which the ack has not been received (\figurename~\ref{proto}).

\begin{figure}[t]
	\fbox{\parbox{\linewidth}{
	\texttt{RCV\_ACK: [receive <forward, ack, i, b>]} \\
	. \qquad \texttt{if b == bit and send[i] == bit:} \\
	. \qquad \qquad \texttt{miss\_send = miss\_send - 1} \\
	. \qquad \qquad \texttt{send[i] = (bit + 1) mod 2} \\
	. \qquad \qquad \texttt{if miss\_send == 0:} \\
	. \qquad \qquad \qquad \texttt{bit = (bit + 1) mod 2} \\
	. \qquad \qquad \qquad \texttt{miss\_rcv = size(rcv)} \\
	\\
	\texttt{RCV\_VAL: [receive <forward,} $w_{jk}^l a_k^{l-1}$\texttt{, k, b>]} \\
	. \qquad \texttt{if b == bit or miss\_send == 0:} \\
	. \qquad \qquad \texttt{send <forward, ack, j, b> to k-th in-neuron} \\
	. \qquad \texttt{if b == bit and rcv[k] == bit:} \\
	. \qquad \qquad \texttt{miss\_rcv = miss\_rcv - 1} \\
	. \qquad \qquad \texttt{rcv[k] = (bit + 1) mod 2} \\
	. \qquad \qquad \texttt{update } $a_j^l$ \texttt{thanks to} $w_{jk}^l a_k^{l-1}$ \\
	. \qquad \qquad \texttt{if miss\_rcv == 0:} \\
	. \qquad \qquad \qquad $\forall$ \texttt{i, send <forward,} $w_{ij}^{l+1} a_j^l$\texttt{, j, b> to i-th out-neuron} \\
	. \qquad \qquad \qquad \texttt{miss\_send = size(send)} \\
	. \qquad \qquad \qquad \texttt{timeout = get\_Time()} \\
	\\
	\texttt{RE-SEND: [miss\_send > 0 and get\_Time() - timeout > DT]} \\
	. \qquad $\forall$ \texttt{i, if send[i] == bit:} \\
	. \qquad \qquad \texttt{send <forward,} $w_{ij}^{l+1} a_j^l$\texttt{, j, b> to i-th out-neuron} \\
	. \qquad \texttt{timeout = get\_Time()} \\
	}}
	\caption{Protocol of communication in the case of feed-forward scenario}
	\label{proto}
\end{figure}

\end{document}